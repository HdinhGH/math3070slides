\documentclass[t,handout]{beamer}
\usepackage{verbatim}
\usepackage{xcolor}
\usepackage{multirow}
\usepackage{amssymb}
\usepackage{tikz}
\usepackage{hyperref}
\usetikzlibrary{positioning,fit}
%\usepackage{enumitem}
\usetheme{Warsaw}
\setbeamertemplate{navigation symbols}{}
\newcommand{\blue}[1]{{\color{blue} #1}}
\newcommand{\red}[1]{{\color{red} #1}}
\newcommand{\grn}[1]{{\color{green} #1}}
\newcommand{\bluRed}[2]{{\color{blue} #1}{\color{red} #2}}
\newcommand{\qtns}[0]{\begin{center} Questions? \end{center}}
\newcommand{\nl}[1]{\vspace{#1 em}}
\newcommand{\cntrImg}[2]{\begin{center}\includegraphics[scale=#2]{#1}\end{center}}
\newcommand{\defn}[1]{{\bf #1}}
\let\emptyset\varnothing
\newcommand{\SampS}[0]{$\mathcal{S}$}

\title{Math 3070, Applied Statistics}

\setlength{\abovedisplayskip}{0pt}
\setlength{\belowdisplayskip}{0pt}
\setlength{\abovedisplayshortskip}{0pt}
\setlength{\belowdisplayshortskip}{0pt}

\begin{document}
\begin{frame}[c]
    \begin{beamercolorbox}[rounded=true,wd=\textwidth,center]{title}
        \usebeamerfont{title}\inserttitle
    \end{beamercolorbox}
    \begin{center}
        Section 1\\
        \nl{0.5}
        November 11, 2019
    \end{center}
\end{frame}
\begin{frame}[c]{Lecture Outline, 11/11}
    Section 8.1
    \begin{itemize}
        \item Z-Tests for Population Mean
    \end{itemize}
\end{frame}

\begin{frame}{$z$ Test for Mean of Normal Distribution with Known $\sigma$}
    \begin{block}{}
    Given a random sample $X_1,\dots,X_n$ from a normal distribution with known standard deviation $\sigma$, the \emph{$z$ test} for the null hypothesis $H_0: \mu=\mu_0$, based on the test statistic $Z=\frac{\overline X-\mu_0}{\sigma/\sqrt{n}}$, is given by the following rejection region, depending on whether a one-tailed or two-tailed test is desired:
    \begin{center}
    \begin{tabular}{ll|l}
    \multicolumn{2}{c}{Alternative hypothesis} & Rejection region \\ \hline
    (Upper-tailed test) & $H_a: \mu>\mu_0$ & $Z\geq z_{\alpha}$ \\
    (Lower-tailed test) & $H_a: \mu<\mu_0$ & $Z\leq -z_{\alpha}$ \\
    (Two-tailed test) & $H_a: \mu\neq\mu_0$ & $|Z|\geq z_{\alpha/2}$\\
    \end{tabular}
    \end{center}
    Here $\alpha$ is the significance level (Type I error probability), and $z_{\alpha}$ is a critical value from the standard normal distribution.
    \end{block}
    \end{frame}
    
    \begin{frame}{Example}
    \begin{block}{}
    A machine is specified to drill holes with diameter 4 mm. We wish to test the null hypothesis $H_0: \mu=4$ against the alternative $H_a: \mu\neq 4$. If the diameters are normally distributed with $\sigma=.2$, and we observe $\overline X=3.87$ in a sample of size 10, do we reject the null hypothesis at the significance level $\alpha=.05$?
    \end{block}
    
    \pause
    The test statistic is 
    $$Z=\frac{\overline X-\mu_0}{\sigma/\sqrt{n}}=\frac{3.87-4}{.2/\sqrt{10}}=-2.06$$
    \pause The rejection region is $\{|Z|>z_{\alpha/2}\}$ where $z_{\alpha/2}=z_{.025}=1.96$.
    
    \vspace{.2cm}
    \pause Since $|Z|=2.06>1.96$, the test statistic is in the rejection region, so we reject the null hypothesis.
    \end{frame}
    
    \begin{frame}{Type II Error Probability for $z$ Test}
    Given a random sample $X_1,\dots,X_n$ from a normal distribution with known standard deviation $\sigma$, the Type II error probability $\beta$ for the $z$ test is as follows:
    \begin{center}
    \renewcommand*{\arraystretch}{1.2}
    \begin{tabular}{l|l}
    Alternative hypothesis & Type II Error Probability $\beta$ \\ \hline
    $H_a: \mu>\mu_0$ & $\Phi(z_\alpha+\frac{\mu_0-\mu}{\sigma/\sqrt{n}})$ \\
    $H_a: \mu<\mu_0$ & $1-\Phi(-z_\alpha+\frac{\mu_0-\mu}{\sigma/\sqrt{n}})$ \\
    $H_a: \mu\neq\mu_0$ & $\Phi(z_{\alpha/2}+\frac{\mu_0-\mu}{\sigma/\sqrt{n}})-\Phi(-z_{\alpha/2}+\frac{\mu_0-\mu}{\sigma/\sqrt{n}})$\\
    \end{tabular}
    \end{center}
    The following formula is the for the sample size so that $\beta$ is no more than a certain value. Note: $\beta$ in this equation is $\beta$ the bound placed on the probability of a type two error.
    $$
    n =
    \begin{cases}
        \bigg[\frac{\sigma(z_{\alpha} + z_\beta)}{\mu_0 - \mu}\bigg]^2 & \text{for one-tailed tests}\\
        \bigg[\frac{\sigma(z_{\alpha/2} + z_\beta)}{\mu_0 - \mu}\bigg]^2& \text{for two-tailed tests}\\
    \end{cases}
    $$
    \end{frame}
    
    \begin{frame}{Example}
    \begin{block}{}
    A machine is specified to drill holes with diameter 4 mm. We wish to test the hypothesis $H_0: \mu=4$ against the alternative $H_a: \mu\neq 4$ using a $z$ test with significance level $\alpha=.05$ on a random sample of size 15. If the diameters are normally distributed with $\sigma=.2$ and the true mean is $\mu=3.95$, what is the Type II error probability?
    \end{block}
    \pause Here $\mu_0=4$, so
    \begin{align*}
    \beta&=
    \Phi(z_{\alpha/2}+\frac{\mu_0-\mu}{\sigma/\sqrt{n}})-\Phi(-z_{\alpha/2}+\frac{\mu_0-\mu}{\sigma/\sqrt{n}})\\
    \uncover<3->{&= \Phi(z_{.025}+\frac{4-3.95}{.2/\sqrt{15}})-\Phi(-z_{.025}+\frac{4-3.95}{.2/\sqrt{15}})\\}
    \uncover<4->{&= \Phi(1.96+ 0.97)-\Phi(-1.96+0.97) \\}
    \uncover<5->{&= \Phi(2.93)-\Phi(-0.99) }
    \uncover<6->{= .837}
    \end{align*}
    \end{frame}
    \begin{frame}{Example}
        \begin{block}{}
            Consider the previous problem. How large should the sample size be so that the probability of a type II error is no more than 0.05?
        \end{block}
        $$z_\beta = -\Phi^{-1}(0.05) \approx 1.645$$
        $$n = \bigg[\frac{\sigma(z_{\alpha/2} + z_\beta)}{\mu_0 - \mu}\bigg]^2 \approx  \bigg[\frac{0.2(1.96 + 1.645)}{4 - 3.95}\bigg]^2 \approx 209$$
        \end{frame}
    \begin{frame}{P-Values for $z$ Test}
        With all of the $z$ test procedures, the P-value may be calculated as follows, where $z$ is the observed value of the test statistic $Z$ (assumed to have a standard normal distribution):
        \begin{center}
        \begin{tabular}{ll|l}
        \multicolumn{2}{c}{Alternative hypothesis} & P-value \\ \hline
        (Upper-tailed test) & $H_a: \mu>\mu_0$ & $P(Z\geq z)$ \\
        (Lower-tailed test) & $H_a: \mu<\mu_0$ & $P(Z\leq z)$ \\
        (Two-tailed test) & $H_a: \mu\neq\mu_0$ & $P(|Z|\geq |z|)$\\
        \end{tabular}
        \end{center}
        
        \end{frame}
        
        \begin{frame}{Example}
        %We return to our previous example to calculate the P-value:
        \begin{block}{}
        A type of candy bar is labeled 60 grams. Someone suggests that the candy bars weigh less than specified. To test this, we gather a random sample of size 25 and observe $\overline X=59.7$ and $\sigma=0.9$ is known. What is the P-value for the test?
        \end{block}
        
        \pause We use the $z$ test with the one-tailed alternative $H_a: \mu<60$. As before, the test statistic  is
        $$T=\frac{\overline X-\mu_0}{\sigma/\sqrt{n}}=\frac{59.7-60}{0.9/\sqrt{25}}\approx -1.667$$
        \pause We use a table to find the probability of observing a value for $T$ at least this extreme:
        $$P=P(T\leq -1.667)
        \uncover<4->{ \approx 0.0478 }$$
        \uncover<5->{ So $P=0.0478$ is the P-value for the test.}
        \end{frame}

\end{document}