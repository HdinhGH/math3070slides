\documentclass[]{beamer}
\usepackage{verbatim}
\usepackage{xcolor}
\usepackage{multirow}
\usepackage{amssymb}
\usepackage{tikz}
\usetikzlibrary{positioning,fit}
%\usepackage{enumitem}
\usetheme{Warsaw}
\setbeamertemplate{navigation symbols}{}
\newcommand{\blue}[1]{{\color{blue} #1}}
\newcommand{\red}[1]{{\color{red} #1}}
\newcommand{\grn}[1]{{\color{green} #1}}
\newcommand{\bluRed}[2]{{\color{blue} #1}{\color{red} #2}}
\newcommand{\qtns}[0]{\begin{center} Questions? \end{center}}
\newcommand{\nl}[1]{\vspace{#1 em}}
\newcommand{\cntrImg}[2]{\begin{center}\includegraphics[scale=#2]{#1}\end{center}}
\newcommand{\defn}[1]{{\bf #1}}
\let\emptyset\varnothing
\newcommand{\SampS}[0]{$\mathcal{S}$}

\title{Math 3070, Applied Statistics}

\begin{document}

\begin{frame}
    \begin{beamercolorbox}[rounded=true,wd=\textwidth,center]{title}
        \usebeamerfont{title}\inserttitle
    \end{beamercolorbox}
    \begin{center}
        Section 1\\
        \nl{0.5}
        September 13, 2019
    \end{center}
\end{frame}

\begin{frame}{Lecture Outline, 9/13}
    Section 3.6
    \begin{itemize}
        \item Poisson Distribution
    \end{itemize}
\end{frame}

\begin{frame}{Poisson Process}
    Consider a process where events occur at random times, such as
    \begin{itemize}
    \item The arrival times of customers at a store
    \item Clicks of a Geiger counter exposed to a radioactive material
    \item Webpage requests on an internet server
    \item Incoming calls to a customer service center
    \end{itemize}
    \pause Such a process is called a \emph{Poisson process} if the following assumptions hold:
    \begin{enumerate}
    \item The mean number of events which occur in a time interval of length $t$ is $\lambda t$, where $\lambda$ is a constant, called the \emph{rate} of the Poisson process.
    \item Events occur only one at a time.
    \item The number of events which occur in a time interval is independent of the number and timing of past events.
    \end{enumerate}
    \end{frame}
    
    \begin{frame}{Poisson Distribution}
    Given a Poisson process with rate $\lambda$, we want to find the pmf of the number of events $X$ which occur in the time interval $I=[0,t]$.
    
    \pause \begin{itemize}
    \item By assumption (1), the mean of $X$ is $\mu=\lambda t$.
    \item Divide the interval $[0,t]$ into equal-width subintervals $I_1, \dots, I_n$, each of length $t/n$.
    %$$I_k = \left[k\cdot\frac t n, (k+1)\cdot\frac t n\right]$$
    \item Let $X_k$ be the number of events which occur in the subinterval $I_k$, so $X=X_1+X_2+\cdots+X_n$. 
    \item By assumption (1), $E(X_k)=\lambda\cdot \frac t n= \frac\mu n$. 
    \item By assumption (2), if $n$ is large, then the probability of more than one event occuring in any given interval $I_k$ is very small. 
    \item So we may approximate $X_k$ as a Bernoulli random variable with parameter $p=\frac{\mu}n$.
    \item By assumption (3), the random variables $X_1,\dots, X_n$ are independent. Thus $X$ is approximately binomial, $Bin(n,\frac \mu n)$.
    \end{itemize}
    \end{frame}
    
    \begin{frame}{Poisson Distribution}
    Recall from calculus that
    $$\lim_{n\to\infty} \left(1+\frac{r}n\right)^n = e^r$$
    We use this to find the pmf of a Poisson random variable $X$. We argued that for large $n$, $X$ is approximately binomial, $Bin(n,\frac{\mu}n)$, so
    \begin{align*}
    P(X=x) & \approx \binom{n}{x} (\mu/n)^x(1-\mu/n)^{n-x} \\
    &= \frac{n(n-1)\cdots(n-x+1)}{x!}(\mu/n)^x(1-\mu/n)^{n-x} \\
    &= \frac{n(n-1)\cdots(n-x+1)}{nn \cdots n}(1-\mu/n)^n(1-\mu/n)^{-x}\cdot \frac{\mu^x}{x!} \\
    &\to 1 \cdot e^{-\mu}\cdot 1\cdot \frac{\mu^x}{x!} \hspace{4cm} (\text{as }n\to\infty)\\
    &= \frac{e^{-\mu}\mu^x}{x!}
    \end{align*}
    \end{frame}
    
    \begin{frame}{Poisson Distribution}
    \begin{block}{}
    Given a Poisson process with rate $\lambda$, the number $X$ of events which occur in a time interval of length $t$ is a \textbf{Poisson} random variable with mean $\mu=\lambda t$. $X\sim Pois(\mu)$ The possible values of $X$ are $0, 1, 2,\dots$
    $$P(X=x)=\frac{e^{-\mu}\mu^x}{x!}$$
    \end{block}
    
    \pause \vspace{.1cm} Example: Suppose that at a small store, customers arrive at an average rate of 6 per hour. What is the probability that during a given hour only 3 or fewer customers will arrive?
    
    \pause \vspace{.3cm}Assuming a Poisson process,
    \begin{align*}
    P(X\leq 3) &= P(X=0)+P(X=1)+P(X=2)+P(X=3) \\
    &= \frac{e^{-6}6^0}{0!} + \frac{e^{-6}6^1}{1!} + \frac{e^{-6}6^2}{2!} + \frac{e^{-6}6^3}{3!} \\
    &= e^{-6} (1+6 + 18 + 36) = 61e^{-6} \approx .151
    \end{align*}
    \end{frame}

    \begin{frame}{Poisson Distribution, Sum of PMF}
    Recall from calculus that
    $$e^\mu=\sum_{x=0}^\infty \frac{\mu^x}{x!}$$
    \pause Therefore, the pmf $f(x)$ of a Poisson random variable satisfies
    \begin{align*}
    \sum_{x=0}^\infty f(x) &= \sum_{x=0}^\infty \frac{e^{-\mu}\mu^x}{x!} \\
    &= e^{-\mu} \sum_{x=0}^\infty \frac{\mu^x}{x!}\\
    &=e^{-\mu}e^\mu\\
    &=1
    \end{align*}
    which shows that $f(x)$ is a valid pmf.
    \end{frame}
    
    \begin{frame}{Mean of Poisson Distribution}
    We may directly calculate the mean of a Poisson random variable $X$ based on the pmf:
    \pause \begin{align*}
    E(X) &= \sum_{x=0}^\infty x\cdot\frac{e^{-\mu}\mu^x}{x!} \\
    &= \sum_{x=1}^\infty \frac{e^{-\mu}\mu^x}{(x-1)!} \\
    &= \sum_{x=0}^\infty \frac{e^{-\mu}\mu^{x+1}}{x!} \\
    &= \mu \sum_{x=0}^\infty \frac{e^{-\mu}\mu^x}{x!} \\
    &= \mu
    \end{align*}
    This is what we expected based on the definition.
    \end{frame}
    
    \begin{frame}{Variance of Poisson Distribution}
    We can also calculate the variance:
    {\small
    \begin{align*}
    E(X^2) &= \sum_{x=0}^\infty x^2\cdot\frac{e^{-\mu}\mu^x}{x!}\\
    &= \sum_{x=1}^\infty x\cdot\frac{e^{-\mu}\mu^x}{(x-1)!}\\
    &= \sum_{x=0}^\infty (x+1)\frac{e^{-\mu}\mu^{x+1}}{x!}\\
    &= \sum_{x=0}^\infty x\frac{e^{-\mu}\mu^{x+1}}{x!} + \sum_{x=0}^\infty \frac{e^{-\mu}\mu^{x+1}}{x!}\\
    &= \sum_{x=1}^\infty \frac{e^{-\mu}\mu^{x+1}}{(x-1)!} + \mu\sum_{x=0}^\infty \frac{e^{-\mu}\mu^{x}}{x!}\\
    &= \sum_{x=0}^\infty \frac{e^{-\mu}\mu^{x+2}}{x!}+\mu
    = \mu^2+\mu
    \end{align*}}
    So $V(X)=E(X^2)-[E(X)]^2=\mu^2+\mu-\mu^2 = \mu$.
    \end{frame}

    \begin{frame}{Example}
        \begin{block}{}
        Suppose that at random times a system suffers breakdowns requiring immediate repairs. If the system breaks down at a rate of once per year, what is the probability that the system will break down 3 or more times in one year?
        \end{block}
        \pause Solution: The given information suggests that the number $X$ of breakdowns in a year is a Poisson random variable with mean $\mu=\lambda t = 1(1)=1$. 
        \pause \begin{align*}
        P(X \geq 3) &= 1 - P(X \leq 2) \\
        &= 1 - P(X=0) - P(X=1)-P(X=2) \\
        &= 1 - \frac{e^{-1}1^0}{0!} - \frac{e^{-1}1^1}{1!} - \frac{e^{-1}1^2}{2!} \\
        &= 1 - e^{-1}\left(1+1+\frac12\right) \\
        &= 1 - \frac{5e^{-1}}2 \approx .080
        \end{align*}
        \end{frame}

    \begin{frame}{Approximating Binomial for Rare Events}
            Binomial random variables with small $p$ and large $n$ can also be approximated with a Poisson random variable, $\mu = np$.
            \pause
            \begin{align*}
                P(X=x) &= \frac{n!}{x!(n-x)!} p^x(1-p)^{n-x}\\
                & = \frac{n(n-1)\cdots (n-x+1)}{x!} \bigg( \frac{\mu}{n} \bigg)^x\bigg( 1 - \frac{\mu}{n} \bigg)^{n-x}\\
                & = \frac{1}{x!}\frac{n^x(1-\frac{1}{n})\cdots(1-\frac{x-1}{n})}{n^x} \mu^x \bigg( 1 - \frac{\mu}{n} \bigg)^{n-x}\\
                & \to \frac{e^{-\mu}\mu^x}{x!}, \quad  (\text{as } n \to \infty)
            \end{align*}
            \pause Note: $n$ and $p$ do not appear as parameters of the exponential distribution, but can be related by $\mu$.
    \end{frame}

    \begin{frame}{Approximating Binomial for Rare Events}
        Consider $E(X)$ and $V(X)$ of $X\sim bin(n,p)$ \\
        as $np \to \mu$ while $n\to \infty$ and $p\to 0$.
        $$E(X) = np \to \mu $$
        $$V(X) = np(1-p) \to \mu $$
        Parameters match in the limit.
    \end{frame}

    \begin{frame}{Example, Approximating Binomial for Rare Events}
        The Prussian army studied the likelihood of a soldier being killed by a horse kick. Over \red{twenty} years, it was observed that 122 soldiers from \blue{10} corps were killed horse kick. The size of each corps is 10,000 soldiers. Approximate the probability that no Prussian soldiers from \grn{two corps} are killed by horse kick in one year.
        \\ \nl{0.5}
        \pause
        Estimate $p$ from the historical data,
        $$ p \approx \frac{122}{10000\cdot \red{20} \cdot \blue{10}}.$$
        Compute $\mu$.
        $$ \mu = \grn{n} p = \grn{2\cdot 10000} \frac{122}{10000 \cdot 20 \cdot 10} = \frac{244}{200} = 1.22$$
        Estimate with $X$ with a $exp(1.22)$ random variable.
        $$ P(X = 0) = \frac{e^{-1.22} 1.22^0}{0!} =e^{-1.22} \approx 0.29523016692 $$
    \end{frame}

    \begin{frame}{Poisson Distribution, Summary}
        \begin{itemize}
            \item $X\sim Pois(\mu)$ mean X is a poisson random variable with rate $\mu$ and its PMF is
            $$P(X=x)=\frac{e^{-\mu}\mu^x}{x!}$$
            \item $$E(X) = V(X) = \mu $$
            \item When modeling number of events in an interval, $\mu = \lambda t$. $\lambda$ is the rate of occurance per unit time. Sometimes, $\mu$ is call the rate instead of $\lambda$. Questions in this class will always assume $\mu$ is the mean.
            \item When approximating the number of rare events in a large population, $\mu = np$. 
        \end{itemize}
    \end{frame}

    \begin{frame}{Midterm September 18, Information}
        \begin{itemize}
            \item There is a midterm on September 18th in class. Calculator and notes allowed.
            \item Study the quizzes, summary slides, and homework. See the Canvas 'Files' tab for information.
            \item Review on September 16th. Come with questions.
            \item Reschedule by Wednesday, September 11, if needed. No makeup or late exams.
            \item One question from this week's material.
            \item No quiz or homework due on exam weeks. Material is shifted to the later week.
        \end{itemize}
    \end{frame}
\end{document}