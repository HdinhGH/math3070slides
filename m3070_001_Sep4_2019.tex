\documentclass[]{beamer}
\usepackage{verbatim}
\usepackage{xcolor}
\usepackage{multirow}
\usepackage{amssymb}
\usepackage{tikz}
\usetikzlibrary{positioning,fit}
%\usepackage{enumitem}
\usetheme{Warsaw}
\setbeamertemplate{navigation symbols}{}
\newcommand{\blue}[1]{{\color{blue} #1}}
\newcommand{\red}[1]{{\color{red} #1}}
\newcommand{\grn}[1]{{\color{green} #1}}
\newcommand{\bluRed}[2]{{\color{blue} #1}{\color{red} #2}}
\newcommand{\qtns}[0]{\begin{center} Questions? \end{center}}
\newcommand{\nl}[1]{\vspace{#1 em}}
\newcommand{\cntrImg}[2]{\begin{center}\includegraphics[scale=#2]{#1}\end{center}}
\newcommand{\defn}[1]{{\bf #1}}
\let\emptyset\varnothing
\newcommand{\SampS}[0]{$\mathcal{S}$}

\title{Math 3070, Applied Statistics}

\begin{document}

\begin{frame}
    \begin{beamercolorbox}[rounded=true,wd=\textwidth,center]{title}
        \usebeamerfont{title}\inserttitle
    \end{beamercolorbox}
    \begin{center}
        Section 1\\
        \nl{0.5}
        September 4, 2019
    \end{center}

\end{frame}

\begin{frame}{Lecture Outline, 9/4}
    Section 3.1
    \begin{itemize}
        \item Random Variables
        \item Probability Mass Functions
        \item Examples
    \end{itemize}
\end{frame}

\begin{frame}{Random Variables}
    \begin{block}{defintion}
    A \text{random variable} is any rule that associates a number with each outcome in of a sample space, $\mathcal{S}$.
    \end{block}
    \begin{block}{notation}
        Most of the time random variables will be denoted with as capital case letters, $X,Y,Z,U,V$.\\
        \nl{0.5}
        When statement such as  $P(X=1)$ are writen this means the probability that the event $X=1$ occurs. To find the probability, one could find the probability of the event which yields $X=1$.
    \end{block}
    \begin{block}{foreshadowing}
        Later we will discuss random variables without explictly referring to a sample space, $\mathcal{S}$. In these cases, one may the values to be outcomes to relate back to the early parts of the course.
    \end{block}
\end{frame}

\begin{frame}{Random Variables, Discrete Versus Continuous}
    \begin{block}{discrete random variables}
        Random variables are \textbf{discrete} if they can take finitely many or countably many values. \textbf{Countably many} means the values can be listed one by one. For example, the integers and whole numbers are countable.
    \end{block}
    \begin{block}{continuous random variables}
        A random variables is \textbf{continuous} when
        \begin{enumerate}
            \item its possible values contain on any interval
            \item and the probability that it take on exactly one value is one, $P(X=c) = 0$ for all $c$.
        \end{enumerate}
        Random variables which take values on anywhere on the number line, an interval or half line, and satisfy property 2 are continuous random variables.
    \end{block}
\end{frame}

\begin{frame}{Discrete Random Variable Example}

Example: Suppose we toss a fair coin 3 times. The set of outcomes is
$$\Omega=\{(TTT),(TTH),(THT),(THH),(HTT),(HTH),(HHT),(HHH)\}$$
Let $X$ be the number of heads. Then $X$ is a random variable:

\begin{center}\begin{tabular}{p{4cm}p{4cm}}
\begin{tabular}{l}
$\begin{aligned}
TTT: X&=0 \\
TTH: X&=1 \\
THT: X&=1 \\
THH: X&=2
\end{aligned}$
\end{tabular} &
\begin{tabular}{l}
$\begin{aligned}
HTT: X&=1 \\
HTH: X&=2 \\
HHT: X&=2 \\
HHH: X&=3
\end{aligned}$
\end{tabular}
\end{tabular}
\end{center}

The possible values of $X$ are 0, 1, 2, and 3.
\end{frame}

\end{document}