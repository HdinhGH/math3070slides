\documentclass[t,handout]{beamer}
\usepackage{verbatim}
\usepackage{xcolor}
\usepackage{multirow}
\usepackage{amssymb}
\usepackage{tikz}
\usepackage{hyperref}
\usetikzlibrary{positioning,fit}
%\usepackage{enumitem}
\usetheme{Warsaw}
\setbeamertemplate{navigation symbols}{}
\newcommand{\blue}[1]{{\color{blue} #1}}
\newcommand{\red}[1]{{\color{red} #1}}
\newcommand{\grn}[1]{{\color{green} #1}}
\newcommand{\bluRed}[2]{{\color{blue} #1}{\color{red} #2}}
\newcommand{\qtns}[0]{\begin{center} Questions? \end{center}}
\newcommand{\nl}[1]{\vspace{#1 em}}
\newcommand{\cntrImg}[2]{\begin{center}\includegraphics[scale=#2]{#1}\end{center}}
\newcommand{\defn}[1]{{\bf #1}}
\let\emptyset\varnothing
\newcommand{\SampS}[0]{$\mathcal{S}$}

\title{Math 3070, Applied Statistics}

\setlength{\abovedisplayskip}{0pt}
\setlength{\belowdisplayskip}{0pt}
\setlength{\abovedisplayshortskip}{0pt}
\setlength{\belowdisplayshortskip}{0pt}

\begin{document}
\begin{frame}[c]
    \begin{beamercolorbox}[rounded=true,wd=\textwidth,center]{title}
        \usebeamerfont{title}\inserttitle
    \end{beamercolorbox}
    \begin{center}
        Section 1\\
        \nl{0.5}
        November 1, 2019
    \end{center}
\end{frame}
\begin{frame}[c]{Lecture Outline, 11/4}
    Section 7.3
    \begin{itemize}
        \item Confidence Intervals with unknown $\mu$ and $\sigma$
    \end{itemize}
\end{frame}
\begin{frame}{Small-Sample Confidence Interval for Mean of Normal}
    Suppose we have a random sample $X_1,\dots,X_n$ from a normal distribution, where the mean $\mu$ and variance $\sigma^2$ are both unknown. 
    \\
    \nl{0.5}
    We have seen that if $n$ is large, then the \emph{pivotal} statistic $T=\frac{\overline{X}-\mu}{S/\sqrt{n}}$ is approximately a standard normal random variable. However, if $n$ is small then $T$ instead has a so-called {\bf t distribution with $\nu=n-1$ degrees of freedom} or $T\sim t(\nu)$.
    \\
    \nl{0.5}
    In this class, if $n<40$, we'll use the $T\sim t(\nu)$. Otherwise, $T\sim N(0,1)$ can be used. In practice, a method should be carefully selected for the data.
\end{frame}
\begin{frame}{T-Distribution}
    We have not studied the t distribution. It has tables and used similiarly to the normal distribution for the purpose computing of finding confidence intervals. 
    
    {\bf Useful Properties:}
    \begin{enumerate}
        \item symmetric about $0$
        \item identified by $\nu$
        \item as $\nu \to \infty$, the distribution becomes $N(0,1)$
        \item bell shaped, but flatten
    \end{enumerate}
    Important to these calculations are critical $t$ values, $t_{\alpha/2,\nu}$:
    $$1-\alpha = P(-t_{\alpha/2,\nu}< T < t_{\alpha/2,\nu})$$
    where $\nu = n-1$ and $T\sim t(\nu)$.\\
    Note:
    $$ \alpha/2 = P(-t_{\alpha/2,\nu}< T) $$
    \end{frame}

    \begin{frame}{Confidence Interval for Mean of Normal}
        Suppose we have a random sample $X_1,\dots,X_n$ from a normal distribution, where the mean $\mu$ and variance $\sigma^2$ are both unknown. 
        
        \pause \vspace{.2cm}Previously, we have used the fact that $n$ is large, then the statistic $\frac{\overline{X}-\mu}{S/\sqrt{n}}$ is approximately equal to $\frac{\overline{X}-\mu}{\sigma/\sqrt{n}}$, which is a standard normal random variable. However, if $n$ is small then this is not a good approximation. Instead, $T$ has a so-called \emph{t distribution}.
        
        \pause \begin{block}{}
        Given a random sample $X_1,\dots,X_n$ from a normal distribution with unknown mean and variance,
        A $100(1-\alpha)\%$ confidence interval for the mean $\mu$ is
        $$\overline{X} \pm \frac{t_{\alpha/2,\nu}\cdot S}{\sqrt{n}}$$
        where $t_{\alpha/2,\nu}$ is a critical value from a \emph{t distribution} with $\nu=n-1$ degrees of freedom.
        \end{block}
        \end{frame}
\begin{frame}{Example}
\begin{block}{}
A process produces alginate beads
with diameters (in mm) normally distributed with unknown mean $\mu$ and unknown standard deviation $\sigma$. A random sample of 9 beads have the following diameters:
$$3.9, 5.1, 5.2, 5.7, 5.8, 6.1, 6.2, 6.3, 6.5$$
Find a 99\% confidence interval for the mean diameter $\mu$.
\end{block}
\pause Here $\alpha=1-.99=.01$, so the relevant critical value is 
$$t_{\alpha/2,\nu}=t_{.005,8} = 3.355$$
\pause The sample mean is $\overline X=5.64$ and the sample standard deviation is $S=.809$, so the confidence interval is given by
$$\overline X \pm \frac{t_{\alpha/2} \cdot S}{\sqrt n} = 5.64 \pm \frac{3.355 \cdot 0.809}{\sqrt{9}} = 5.64 \pm  0.90$$
\end{frame}

\begin{frame}{Prediction Interval for a Normal Population}
    Given a random sample $X_1,\dots,X_n$ from a normal distribution, suppose we want to construct an interval $[A,B]$ which we can be 95\% confident will contain a future observation $X_{n+1}$. Such an interval is called a \emph{prediction interval}.
    
    \vspace{.2cm}\pause The statistic $T=\frac{\overline X-X_{n+1}}{S\sqrt{1+\frac1n}}$ has a t distribution with $\nu=n-1$ degrees of freedom.
    \begin{block}{}
    Given a random sample $X_1,\dots,X_n$ from a normal distribution, a $100(1-\alpha)\%$ prediction interval for an independent observation $X_{n+1}$ is
    $$\overline X \pm t_{\alpha/2,n-1}\cdot S\sqrt{1+\frac1n}$$
    \end{block}
    \end{frame}
    
    \begin{frame}{Example}
    \begin{block}{}
    An article reports the following data on the breakdown voltage of electrically stressed circuits, assumed to be normally distributed:
    \begin{align*}
    &1470, 1510, 1690, 1740, 1900, 2000, 2030, 2100, 2200, \\
    & 2290, 2380, 2390, 2480, 2500, 2580, 2190, 2700
    \end{align*}
    Find a 95\% confidence interval for the mean $\mu$. Then find a 95\% prediction interval for a future observation.
    \end{block}
    
    We found $\overline X=2126.5$ and $S^2=137324.3$.
    The critical value is $t_{\alpha/2,\nu}=t_{.025,16}=2.120$, giving the confidence interval for $\mu$:
    $$\overline{X} \pm \frac{t_{\alpha/2,\nu}\cdot S}{\sqrt{n}} = 2126.5 \pm 190.5$$
    Likewise we get the prediction interval:
    $$\overline X \pm t_{\alpha/2,\nu}\cdot S\sqrt{1+\frac1n} = 2126.5 \pm 808.4$$
    \end{frame}

\end{document}