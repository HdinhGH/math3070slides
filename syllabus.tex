\documentclass[12pt]{article}
\usepackage[height=8.5in]{geometry}
\usepackage{url}
\setlength{\parindent}{0in}
\setlength{\parskip}{\baselineskip}
\pagestyle{empty}
\begin{document}
\begin{center}
{\bf MATH 3070, Spring 2013: Applied Statistics I}
\end{center}

\vspace{-.5cm}
{\bf Class:} MWF 9:40-10:30am, LCB 219\\
{\bf Instructor:}  Brent Kerby\\
{\bf Office Hour}: Mondays 3:00-4:00pm, JWB 214\\
{\bf Email}: kerby@math.utah.edu

{\bf Text:}  {\sl Probability \& Statistics for Engineering and the Sciences, $8^{th}$ edition}, by Jay L. 
Devore.

{\bf Content:} The course will cover the material in chapters 1 to 9 of the textbook. This includes simple descriptive statistics (mean, median, standard deviation, histograms, boxplots, etc.), basic probability theory (including the binomial, geometric, Poisson, normal, and exponential distributions), the Central Limit Theorem, maximum-likelihood estimation, confidence intervals, and hypothesis testing.

{\bf Lab:} Students will learn the statistical software package R in a weekly lab section. Students \emph{must pass} the lab section in order to pass the course.

{\bf Quizzes:} Occasional quizzes may be given on material covered previously in class. These will be weighted in the grading as the equivalent of a homework assignment.

{\bf Homework:} Homework is assigned from Devore and will be collected in class on Wednesdays, on the dates shown in the schedule. Although students are expected to be able to do all of the exercises, only those in the column ``Graded exercises" are to be turned in. In order to receive credit, students must clearly show the steps of their solution. Homework will be accepted up to one class period late with no penalty. Late homework will not be accepted beyond that point; however, out of the homework and quizzes, the lowest two scores will be dropped. \emph{Please carefully read the policy on collaboration, on the back of this syllabus.}

{\bf Extra credit:} Occasionally extra credit problems will be announced in class. These are due at the same time as the corresponding homework (with the same late policy) and should be turned in as part of the homework.

{\bf Exams:}  There will be two in-class midterm exams, scheduled for February 20 and April 10, and there will be a comprehensive final exam on May 1 from 8:00-10:00am, also in our classroom. Books and notes may \emph{not} be used during the exams. Students will need to bring a scientific or graphing calculator to use during exams.

\newpage{\bf Grades:} Grades and other course materials may be viewed online through Canvas (accessible via CIS: \url{http://cis.utah.edu}).
Grades will be determined as follows:

\begin{tabular}{l|r}
\begin{tabular}{lr}
Lab & 10\% \\
Homework \& Quizzes & 20\% \\
Midterm Exams & 40\% \\
Final Exam & 30\%
\end{tabular}
&
\begin{tabular}{l|r}
\begin{tabular}{lr}
A & 93+ \\
A- & 90-93 \\
B+ & 87-90 \\
B & 83-87 \\
B- & 80-83 \\
C+ & 77-80 \\
\end{tabular}
&
\begin{tabular}{lr}
C & 73-77 \\
C- & 70-73 \\
D+ & 67-70 \\
D & 63-67 \\
D- & 60-63 \\
E & 60- 
\end{tabular}
\end{tabular}
\end{tabular}

{\bf Collaboration:} 
The graded homework is designed to measure students' progress and to provide meaningful feedback to help students master the material. These purposes are not well-served if students rely on others while writing their solutions. Therefore, \emph{each student is expected to write up his or her own solution without reference to anyone else's solution.} If two assignments are found to contain unusually similar solutions, then both assignments will receive at most half credit. 
For this class, examples of \emph{unacceptable} collaboration include
\begin{itemize}
\item Working a problem as a group at a blackboard and then using the writing on the board as a basis for your solution
\item Borrowing someone else's solution to use as a reference in writing your own
\item Helping a friend by lending them your solution
\item Trying a problem yourself, then giving up and copying someone else's answer
\end{itemize}
Examples of acceptable collaboration include
\begin{itemize}
\item Working through a practice exercise together (i.e., \emph{not} a graded exercise)
\item Checking your answers with another student
\item Helping a friend by pointing them to the relevant formulas
\item Discussing and clarifying the concepts behind an exercise
\end{itemize}


{\bf ADA Statement:} The University of Utah seeks to provide equal access to its programs, services, and activities for people with disabilities. If you will need accommodations in the class, reasonable prior notice needs to be given to the Center for Disability Services, 162 Union Building, 581-5020 (V/TDD). CDS will work with you and the instructor to make arrangements for accommodations.

\end{document}

